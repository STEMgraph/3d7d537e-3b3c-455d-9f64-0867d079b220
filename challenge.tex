\begin{challenge}
    \chatitle{Fixed-Size Containers with std::array in C++}
    \begin{chadescription}
    In C++, the \texttt{std::array} class provides a modern, safer alternative to traditional C-style arrays. 
    Unlike raw arrays, \texttt{std::array} is a fixed-size container that comes with additional functionality, such as bounds checking and a rich set of member functions.
    It combines the efficiency of arrays with the features of object-oriented programming, making it a powerful tool for managing fixed-size collections of data.

    At its core, \texttt{std::array} helps developers write cleaner and more maintainable code. 
    With built-in support for operations like size retrieval and range-based iteration, it reduces the need for manual bookkeeping and error-prone practices often associated with raw arrays.
    Furthermore, since \texttt{std::array} is part of the Standard Template Library (STL), it integrates seamlessly with other modern C++ features, such as algorithms and iterators.

    In this challenge, you'll explore how to declare, initialize, and use \texttt{std::array} in practical scenarios.
    You'll compare it to traditional C-style arrays to understand the added safety and functionality it provides.
    By the end, you’ll have a solid grasp of when and why to use \texttt{std::array} in your C++ programs.
    Let’s dive in and see how \texttt{std::array} can simplify and enhance your code!
    \end{chadescription}

    \begin{task}
        Write a simple program using \texttt{std::array} that performs the following steps:
        \begin{enumerate}
            \item Declare an \texttt{std::array} of size 5 to store integer values.
            \item Initialize the array with the values \texttt{1, 2, 3, 4, 5}.
            \item Use a loop to print each element of the array to the terminal.
            \item Modify the third element of the array to the value \texttt{10}.
            \item Print the updated array to the terminal.
        \end{enumerate}
        
        Save your program and run it to ensure the output matches your expectations.
        
        \begin{questions}
            \item What changes are required to modify a specific element in an \texttt{std::array}? How does this differ from modifying an element in a C-style array?
            \item Why is the size of the array fixed when using \texttt{std::array}? What happens if you try to exceed the declared size?
            \item How does the \texttt{size()} method of \texttt{std::array} simplify working with arrays compared to using \texttt{sizeof} in C?
        \end{questions}
    \end{task}


    \begin{task}
        Write a program using \texttt{std::array} that performs the following steps:
        \begin{enumerate}
            \item Declare and initialize an \texttt{std::array} of size 10 with the values \texttt{2, 4, 6, 8, 10, 12, 14, 16, 18, 20}.
            \item Implement a function named \texttt{calculateSum} that takes the \texttt{std::array} as a parameter and returns the sum of its elements.
            \item Use a range-based \texttt{for} loop to print all even numbers in the array to the terminal.
            \item Reverse the elements of the array manually by swapping elements from the beginning and the end.
            \item Print the reversed array, followed by the sum of its elements, to the terminal.
        \end{enumerate}
        
        Save your program, compile it, and ensure that it runs correctly. Think carefully about how \texttt{std::array} makes your implementation easier and more robust compared to a raw C-style array.
        
        \begin{questions}
            \item How does using \texttt{std::array} simplify the implementation of a fixed-size collection compared to using a C-style array?
            \item When reversing the array manually, what challenges might arise with raw arrays that \texttt{std::array} helps avoid?
            \item What benefits does the range-based \texttt{for} loop offer in printing the array's elements?
        \end{questions}
    \end{task}
        

    \begin{task}
        Write a program using \texttt{std::array} that performs the following steps:
        \begin{enumerate}
            \item Declare and initialize an \texttt{std::array} of size 7 with the values \texttt{3, 1, 4, 1, 5, 9, 2}.
            \item Implement a function named \texttt{findMax} that takes the \texttt{std::array} as a parameter and returns the largest element in the array.
            \item Implement a function named \texttt{rotateRight} that takes the \texttt{std::array} as a parameter and shifts all elements one position to the right, moving the last element to the first position.
            \item Call \texttt{rotateRight} three times on the array, printing the array to the terminal after each rotation.
            \item Print the largest element in the final rotated array using the \texttt{findMax} function.
        \end{enumerate}
        
        Save your program, compile it, and ensure it runs correctly. This task requires you to think carefully about how to manipulate the contents of the \texttt{std::array} and handle array indices correctly.
        
        \begin{questions}
            \item How does the implementation of \texttt{rotateRight} highlight the benefits of using \texttt{std::array} over a raw C-style array?
            \item What challenges might arise when accessing array elements during the rotation, and how did you address these challenges?
            \item Why is it important to test the \texttt{findMax} function with various inputs? How would you modify the program to handle an empty array safely?
        \end{questions}
    \end{task}
        
    \begin{advise}
        Working with \texttt{std::array} helps you understand the balance between safety and performance in C++. 
        Unlike raw C-style arrays, \texttt{std::array} provides clear and consistent methods to work with fixed-size collections, making your code more readable and less prone to errors. 
        Through this challenge, you have explored key concepts such as passing arrays to functions, iterating over elements efficiently, and manipulating the contents of an array. 
        Always think carefully about how your operations on arrays affect memory and program logic, especially when dealing with indices and boundaries. 
        With practice, you'll find \texttt{std::array} to be a reliable and powerful tool for organizing and processing data in C++ programs.
    \end{advise}
        
\end{challenge}
    
